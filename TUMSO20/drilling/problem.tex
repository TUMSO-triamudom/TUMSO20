\documentclass[11pt,a4paper]{article}

\newcommand{\tumsoTime}{09:00 น. - 12:00 น.}
\newcommand{\tumsoRound}{1}

\usepackage{../tumso}

\begin{document}

\begin{problem}{ฝึกฝนทำโจทย์ (Drilling)}{standard input}{standard output}{1 seconds}{16 megabytes}{150}

Blackslex เป็น Student อยู่แถว ๆ นี้ ที่ไม่ได้ออกโจทย์ข้อนี้ (อ้าว โดนเกียน)

ก่อนการแข่งขัน 20th Tough Universe Military Struggle Olympic (TUMSO 20) เขาต้องการฝึกทำโจทย์ทั้งหมด $N$ ข้อ แต่เขาก็ไม่รู้ว่าจะทำข้อไหนดี เขาจึงไปถาม icy ยามประจำเว็บไซต์เว็บหนึ่งว่าเขาควรทำโจทย์ข้อไหนดี

\begin{center}
    \includegraphics[width=8cm]{drilling/image.png} \\
    (ภาพ: Blackslex และ ยามประจำเว็บไซต์เว็บหนึ่ง)
\end{center}

icy ก็ให้โจทย์ Blackslex มาทั้งหมด $N$ ข้อ โดยแต่ละข้อมีความยาก $A_{i}$ แต่เขามีเงื่อนไขของการทำโจทย์พวกนี้ คือ Blackslex ต้องทำเรียงข้อไม่สามารถข้ามทำข้อที่ง่ายกว่าก่อนได้ กล่าวคือ หากเขาต้องการทำโจทย์ข้อที่ $i$ เขาต้องทำโจทยข้อ $j$ ไปแล้วทุกข้อ เมื่อ $1\leq j<i$ 

หลังจาก Blackslex ได้รับโจทย์มาเขาก็ตระหนักได้ว่าถ้าในแต่ละเขาทำโจทย์เยอะเกินไปก็อาจจะเหนื่อย เขาจึงต้องการทำโจทย์ทำโจทย์\textbf{ทั้งหมด} $N$ ข้อนี้ในเวลา $K$ วันพอดี (ห้ามขาดห้ามเกิน) ในแต่ละวันเขาจะได้รับ \textit{ความเหนื่อย} จากโจทย์ที่เขาทำเป็นความยากของโจทย์ที่ยากที่สุดที่เขาทำในวันนั้น ๆ

เนื่องจาก Blackslex ไม่อยากเหนื่อยมาก เขาจึงต้องการให้\textbf{คุณ}ช่วยเขาคำนวณผลรวมของความเหนื่อยที่น้อยที่สุดที่เขาจะทำโจทย์ได้ทั้งหมดที่เป็นไปได้ให้เขา

\InputFile
ข้อมูลนำเข้ามีทั้งหมด $2$ บรรทัด

บรรทัดแรกประกอบด้วยจำนวนเต็ม $N$ และ $K$ แทนจำนวนโจทย์ที่ต้องการฝึก และจำนวนวันที่ทำโจทย์ $(1\leq N\leq 5\cdot 10^{5},1\leq K\leq \min\{N,100\})$

บรรทัดที่ $2$ ประกอบด้วยจำนวนเต็ม $N$ จำนวน คือ $A_1,A_2,\ldots,A_N$ โดยที่ $A_i$ แทนความยากของโจทย์ข้อที่ $i$ $(1\leq A_i\leq 10^{10})$

\OutputFile
ตอบจำนวนเต็มเพียงหนึ่งตัว แทนผลรวมของความเหนื่อยที่น้อยที่สุดที่เป็นไปได้

\Scoring
ชุดทดสอบจะถูกแบ่งเป็น 6 ชุด จะได้คะแนนในแต่ละชุดก็ต่อเมื่อโปรแกรมให้ผลลัพธ์ถูกต้องในชุดทดสอบย่อยทั้งหมด

\begin{description}

\item[ชุดที่ 1 (6 คะแนน)] จะมี $1\leq N\leq 10$ และ $N=K$
\item[ชุดที่ 2 (4 คะแนน)] จะมี $A_i=1$
\item[ชุดที่ 3 (15 คะแนน)] จะมี $A_1\leq A_2\leq\cdots\leq A_N$
\item[ชุดที่ 4 (46 คะแนน)] จะมี $1\leq N\leq 10^3$
\item[ชุดที่ 4 (31 คะแนน)] จะมี $1\leq N\leq 5\cdot 10^4$ และ $1\leq K\leq\min\{N,20\}$
\item[ชุดที่ 6 (48 คะแนน)] ไม่มีเงื่อนไขเพิ่มเติม

\end{description}

\Examples

\begin{example}
\exmp{6 2
6 5 4 3 2 1
}{7
}%
\exmp{3 2
7 6 2
}{9
}%
\end{example}

\Note

คำอธิบายตัวอย่างที่ 1 \\
Blackslex ทำโจทย์ทั้งหมด $2$ วัน
\begin{itemize}
    \item วันที่ $1$ ทำโจทย์ $[6,5,4,3,2]$ ทำให้ Blackslex มีความเหนื่อย $6$ หน่วย
    \item วันที่ $2$ ทำโจทย์ $[1]$ ทำให้ Blackslex มีความเหนื่อย $1$ หน่วย
\end{itemize}
รวมแล้ว Blackslex จะมีความเหนื่อย $6+1=7$ หน่วย

คำอธิบายตัวอย่างที่ 2 \\
Blackslex ทำโจทย์ทั้งหมด $2$ วัน
\begin{itemize}
    \item วันที่ $1$ ทำโจทย์ $[7,6]$ ทำให้ Blackslex มีความเหนื่อย $7$ หน่วย
    \item วันที่ $2$ ทำโจทย์ $[2]$ ทำให้ Blackslex มีความเหนื่อย $2$ หน่วย
\end{itemize}
รวมแล้ว Blackslex จะมีความเหนื่อย $7+2=9$ หน่วย

\end{problem}

\end{document}
