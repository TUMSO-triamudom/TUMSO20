\documentclass[11pt,a4paper]{article}

\newcommand{\tumsoTime}{09:00 น. - 12:00 น.}
\newcommand{\tumsoRound}{1}

\usepackage{../tumso}

\begin{document}

\begin{problem}{เพนกวินตกค่าย}{standard input}{standard output}{1 seconds}{128 megabytes}{100}

% เดี๋ยวแก้ต่อ :D

ใน 20th Tough Universe Military Struggle Olympic (TUMSO 20) เพนกวินผู้เข้าแข่งขันที่แข่งมาหลายต่อหมายรอบ ซึ่งรูปแบบการแข่งขันจะเป็นการหาจำนวนรูปแบบของแต้มที่เป็นไปได้ที่แตกต่างกันที่ทหารแต่ละคนจะได้ในแต่ละรอบ โดยที่ทหารแต่ละคนสามารถเลือกได้ว่าจะนอนหันซ้ายหรือขวา โดยที่แต้มที่ทหารแต่ละคนจะได้คือจำนวนทหารที่สบตาโดยสมมติว่าทหารสามารถมองทะลุกันได้ โดยจะให้หาทั้งหมด $Q$ รอบ โดยแต่ละรอบจะมีทหารทั้งหมด $N$ คน

ตัวอย่างเช่น $\rightarrow\leftarrow\leftarrow$ แต้มที่ได้จะเป็น $2$ $1$ $1$

\InputFile
ข้อมูลนำเข้ามีทั้งหมด $Q+1$ บรรทัด

บรรทัดแรกประกอบด้วยจำนวนเต็ม $Q$ แทนจำนวนคำถาม $(1\leq Q\leq 10^{6})$

บรรทัดที่ $2$ ถึง $Q+1$ ประกอบด้วยจำนวนเต็ม $N_i$ แทนจำนวนทหารในคำถามที่ $i$ $(1\leq N\leq 10^{18})$

\OutputFile
มีทั้งหมด $Q$ บรรทัด ซึ่งบรรทัดที่ $i$ แสดงถึงเศษจากการหารคำตอบของคำถามที่ $i$ ด้วย $998244353$

\Scoring
ชุดทดสอบจะถูกแบ่งเป็น 6 ชุด จะได้คะแนนในแต่ละชุดก็ต่อเมื่อโปรแกรมให้ผลลัพธ์ถูกต้องในชุดทดสอบย่อยทั้งหมด

\begin{description}

\item[ชุดที่ 1 (3 คะแนน)] จะมี $1\leq N\leq 10$
\item[ชุดที่ 2 (5 คะแนน)] จะมี $1\leq N\leq 20$
\item[ชุดที่ 3 (7 คะแนน)] จะมี $1\leq N\leq 1000$
\item[ชุดที่ 4 (11 คะแนน)] จะมี $1\leq N\leq 10^6$
\item[ชุดที่ 5 (13 คะแนน)] จะมี $1\leq N\leq 10^9$
\item[ชุดที่ 6 (61 คะแนน)] ไม่มีเงื่อนไขเพิ่มเติม 

\end{description}

\Examples

\begin{example}
\exmp{2
2
100
}{2
882499618
}%
\end{example}

\Note

คำอธิบายตัวอย่างที่ 1

คำอธิบายคำถามที่ 1 \\
การนอนหันของทหารเป็นไปได้ $4$ แบบดังนี้
\begin{enumerate}
    \item $\leftarrow\leftarrow$ ได้แต้ม $0$ $0$
    \item $\leftarrow\rightarrow$ ได้แต้ม $0$ $0$
    \item $\rightarrow\leftarrow$ ได้แต้ม $1$ $1$
    \item $\rightarrow\rightarrow$ ได้แต้ม $0$ $0$ 
\end{enumerate}
จะได้ว่ามีรูปแบบของแต้มที่ทหารแต่ละคนจะได้ทั้งหมด $2$ แบบ ได้แก่ $0$ $0$ และ $1$ $1$

คำอธิบายคำถามที่ 2 \\
ในทำนองเดียวกันกับคำถามที่ 1 จะได้ว่ามีเศษจากการหารจำนวนรูปแบบของแต้มที่ทหารแต่ละคนจะได้ทั้งหมดด้วย $998244353$ ได้ $882499618$

\end{problem}

\end{document}
