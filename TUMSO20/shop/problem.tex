\documentclass[11pt,a4paper]{article}

\newcommand{\tumsoTime}{09:00 น. - 12:00 น.}
\newcommand{\tumsoRound}{1}

\usepackage{../tumso}

\begin{document}

\begin{problem}{Shop 2}{standard input}{standard output}{1 seconds}{512 megabytes}{250}

% เดี๋ยวแก้ต่อ :D

ร้านค้าแห่งหนึ่งจะนำเข้าสินค้าทั้งหมด $N$ ชนิด ในเวลา $N$ วัน โดยในวันที่ $i$ จะนำเข้าสินค้าชนิดที่ $A_i$ (หรือเรียกอีกอย่างว่าตารางนำเข้าสินค้า) เสธวินต้องการจะไปซื้อของที่ร้านนี้ แต่เนื่องจาก เสธวินเรื่องมากจึงมีเงื่อนไขอยู่ว่าเสธวินไปร้านค้าในวันที่ $d$ ก็ต่อเมื่อร้านค้าได้นำเข้าสินค้าชนิดที่ $1$ ถึง $d$ แล้วเท่านั้น แต่เนื่องจากร้านค้านั้นชอบเกียนโดยการเปลี่ยนแปลงตารางนำเข้าสินค้าเล่น โดยจะมีการเปลี่ยนแปลงทั้งหมด $Q$ รอบโดยจะมีวิธีการเปลี่ยนแปลงทั้งหมด 2 ประเภท ได้แก่

\begin{itemize}
    \item $T_i=1$ : สลับ $A_{L_i}$ กับ $A_{R_i}$
    \item $T_i=2$ : เปลี่ยนช่วง $A_{L_i},A_{L_i+1},\ldots,A_{R_i}$ เป็น $A_{R_i},A_{R_i-1},\ldots,A_{L_i}$
\end{itemize}

เสธวินต้องการจะทราบว่าก่อนมีการเปลี่ยนครั้งแรก และหลังจากมีการเปลี่ยนแปลงแต่ละครั้ง จะมีกี่วันที่เสธวินสามารถไปร้านได้

\InputFile
ข้อมูลนำเข้ามีทั้งหมด $Q+2$ บรรทัด

บรรทัดแรกประกอบด้วยจำนวนเต็ม $N$ และ $Q$ แทนจำนวนประเภทของสินค้า และจำนวนครั้งที่ร้านค้าจะกระทำการเกียน $(2\leq N\leq 10^{5},1\leq Q\leq 10^5)$

บรรทัดที่ $2$ ประกอบด้วยจำนวนเต็ม $N$ จำนวน คือ $A_1,A_2,\ldots,A_N$ โดยที่ $A_i$ แทนประเภทของสินค้าชิ้นที่ $i$ $(1\leq A_i\leq N)$ และ $A_i \neq A_j$ สำหรับทุก $i \neq j$

อีก $Q$ บรรทัดประกอบด้วยจำนวนเต็ม $3$ จำนวน คือ $T_i,L_i$ และ $R_i$ แทนประเภทการเปลี่ยนแปลง และ $L_i$ กับ $R_i$ ในการประเภทของการเปลี่ยนแปลง $T_i$ $(T_i\in\{1,2\},1\leq L_i<R_i\leq N)$

\OutputFile
มีทั้งหมด $Q+1$ บรรทัด

บรรทัดแรกประกอบไปด้วยจำนวนเต็มหนึ่งจำนวน แทนจำนวนวันที่เสธวินสามารถไปร้านได้ก่อนจะมีการเปลี่ยนแปลง

บรรทัดที่ $2$ ถึง $Q+1$ ประกอบด้วยจำนวนเต็มหนึ่งจำนวน แทนจำนวนวันที่เสธวินสามารถไปได้หลังจากมีการเปลี่ยนแปลงครั้งที่ $i$ โดยที่ $1\leq i\leq Q$

\Scoring
ชุดทดสอบจะถูกแบ่งเป็น 2 ชุด จะได้คะแนนในแต่ละชุดก็ต่อเมื่อโปรแกรมให้ผลลัพธ์ถูกต้องในชุดทดสอบย่อยทั้งหมด

\begin{description}

\item[ชุดที่ 1 (13 คะแนน)] จะมี $2\leq N\leq 500,1\leq Q\leq 500$
\item[ชุดที่ 2 (23 คะแนน)] จะมี $2\leq N\leq 5000,1\leq Q\leq 5000$
\item[ชุดที่ 3 (38 คะแนน)] จะมี $R_i-L_i\leq 10$
\item[ชุดที่ 4 (88 คะแนน)] จะมี $T_i=1$
\item[ชุดที่ 5 (88 คะแนน)] ไม่มีเงื่อนไขเพิ่มเติม

\end{description}

\Examples

\begin{example}
\exmp{7 2
3 1 2 5 4 6 7
2 1 3
1 1 2
}{4
5
6
}%
\end{example}

\Note

ก่อนมีการเปลี่ยนแปลง ร้านที่เสธวินสามารถไปได้มีทั้งหมด $4$ ร้าน ได้แก่ร้านที่ $3$, $5$, $6$ และ $7$

หลังจากมีการเปลี่ยนแปลงครั้งที่ $1$ ประเภทของสินค้าในแต่ละวันได้แก่ $2$ $1$ $3$ $5$ $4$ $6$ $7$ ดังนั้นร้านที่เสธวินสามารถไปได้มีทั้งหมด $5$ ร้าน ได้แก่ร้านที่ $2$, $3$, $5$, $6$ และ $7$

หลังจากมีการเปลี่ยนแปลงครั้งที่ $2$ ประเภทของสินค้าในแต่ละวันได้แก่ $1$ $2$ $3$ $5$ $4$ $6$ $7$ ดังนั้นร้านที่เสธวินสามารถไปได้มีทั้งหมด $6$ ร้าน ได้แก่ร้านที่ $1$, $2$, $3$, $5$, $6$ และ $7$

\end{problem}

\end{document}
