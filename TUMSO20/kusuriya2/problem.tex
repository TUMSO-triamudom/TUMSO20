\documentclass[11pt,a4paper]{article}

\newcommand{\tumsoTime}{09:00 น. - 12:00 น.}
\newcommand{\tumsoRound}{1}

\usepackage{../tumso}

\begin{document}

\begin{problem}{Kusuriya 2}{standard input}{standard output}{0.25 seconds}{256 megabytes}{75}

หลังจากที่ Maomao เลือกสมุนไพรเสร็จแล้วขั้นตอนต่อมาก็คือขั้นตอนการปรุง แต่หากใช้เพียงแค่สมุนไพรอย่างเดียว รสชาติที่ออกมานั้นอาจจะไม่ถูกใจมากนัก ดังนั้นจึงต้องใช้เครื่องปรุงต่าง ๆ มาช่วย

\begin{center}
    \includegraphics[width=8cm]{kusuriya2/image0.jpg}
\end{center}

ซึ่ง Maomao มีเครื่องปรุงทั้งหมด $N$ ชนิด โดยที่ชนิดที่ $i$ จะมีความหวาน $A_i$ และความเปรี้ยว $B_i$ และเมื่อนำแต่ละเครื่องปรุงมารวมกัน ความหวานลัพธ์เท่ากับผลคูณของความหวานของเครื่องปรุงทุกชนิดที่ใช้ ความเปรี้ยวลัพธ์เท่ากับผลบวกของความเปรี้ยวของเครื่องปรุงทุกชนิดที่ใช้ ค่ารสชาติเท่ากับผลต่างของความหวานลัพธ์และความเปรี้ยวลัพธ์ (ต้องใช้เครื่องปรุงอย่างน้อย 1 ชนิด เพื่อให้รสชาติที่ออกมานั้นถูกใจบ้าง) หรืออธิบายอีกอย่างได้ว่า \\
ให้ $S\in\mathcal{P}(\{1,2,\ldots,N\})-\{\varnothing\}$ แทนเซ็ตของชนิดเครื่องปรุงที่นำมาใช้ ให้ $f:\mathcal{P}(\{1,2,\ldots,N\})\to\mathbb{N}$ แทนความหวานลัพธ์ของเครื่องปรุงทุกชนิดที่ใช้ จะได้ว่า \[f(S)=\prod\limits_{i\in S}A_i\] 
และให้ $g:\mathcal{P}(\{1,2,\ldots,N\})\to\mathbb{N}$ แทนความเปรี้ยวลัพธ์ของเครื่องปรุงทุกชนิดที่ใช้ จะได้ว่า \[g(S)=\sum\limits_{i\in S}B_i\]
จะได้ว่า $h:\mathcal{P}(\{1,2,\ldots,N\})\to\mathbb{N}$ แทนค่ารสชาติของเครื่องปรุงทุกชนิดที่ใช้ จะได้ว่า \[h(S)=\left|\prod\limits_{i\in S}A_i-\sum\limits_{i\in S}B_i\right|=\left|f(S)-g(S)\right|\]
เพราะฉะนั้นจะได้ว่า $l:\mathcal{P}(\{1,2,\ldots,N\})\to\mathbb{N}$ แทนค่าความอร่อยของเครื่องปรุงทุกชนิดที่ใช้ จะได้ว่า \[l(S)=\sum\limits_{x=1}^{h(S)}x^3-\sum\limits_{x=1}^{h(S)}x^2+\sum\limits_{x=1}^{h(S)}x\]
แต่เนื่องจาก Maomao นั้นชอบกินพิษ จึงอยากทำยาที่มีความอร่อยน้อยที่สุดที่เป็นไปได้ Maomao จึงอยากให้คุณช่วยว่าความอร่อยที่น้อยที่สุดเป็นเท่าใด

\InputFile
ข้อมูลนำเข้ามีทั้งหมด $N+1$ บรรทัด

บรรทัดแรกประกอบด้วยจำนวนเต็ม $N$ แทนจำนวนเครื่องปรุงที่สามารถใช้ได้ $(1\leq N\leq 10)$

บรรทัดที่ $2$ ถึง $N+1$ แต่ละบรรทัดประกอบด้วยจำนวนเต็ม $2$ จำนวน คือ $A_i$ และ $B_i$ แทนความหวานและความเปรี้ยวของเครื่องปรุงชนิดที่ $i$ $(1\leq A_i,B_i\leq 100)$

รับประกันว่าถ้า $S=\{1,2,\ldots,N\}$ แล้ว $f(S)$ และ $g(S)$ จะมีค่าไม่เกิน $5\cdot 10^4$

\OutputFile
ตอบจำนวนเต็มเพียงหนึ่งตัว แทนความอร่อยที่น้อยที่สุดที่เป็นไปได้

\Scoring
ชุดทดสอบจะถูกแบ่งเป็น 5 ชุด จะได้คะแนนในแต่ละชุดก็ต่อเมื่อโปรแกรมให้ผลลัพธ์ถูกต้องในชุดทดสอบย่อยทั้งหมด

\begin{description}

\item[ชุดที่ 1 (2 คะแนน)] จะมี $N=1$
\item[ชุดที่ 2 (3 คะแนน)] จะมี $A_i=1$ และ $B_i=1$ สำหรับทุก $i=1,2,\ldots,N$
\item[ชุดที่ 3 (3 คะแนน)] จะมี $N=2$
\item[ชุดที่ 4 (11 คะแนน)] จะมี $1\leq N\leq 5$
\item[ชุดที่ 5 (56 คะแนน)] ไม่มีเงื่อนไขเพิ่มเติม

\end{description}

\Examples

\begin{example}
\exmp{2
3 8
5 8
}{1
}%
\end{example}

\Note

เลือก $S=\{1,2\}$ จะได้ว่า $f(S)=\prod\limits_{i\in S}A_i=A_1\cdot A_2=15$ และ $g(S)=\sum\limits_{i\in S}B_i=B_1+B_2=16$ \\เพราะฉะนั้น $h(S)=\left|f(S)-g(S)\right|=1$ ดังนั้น $l(S)=\sum\limits_{x=1}^{h(S)}x^3-\sum\limits_{x=1}^{h(S)}x^2+\sum\limits_{x=1}^{h(S)}x=1$

\end{problem}

\end{document}
