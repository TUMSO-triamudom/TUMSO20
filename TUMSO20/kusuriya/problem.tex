\documentclass[11pt,a4paper]{article}

\newcommand{\tumsoTime}{09:00 น. - 12:00 น.}
\newcommand{\tumsoRound}{1}

\usepackage{../tumso}

\begin{document}

\begin{problem}{Kusuriya}{standard input}{standard output}{1 seconds}{256 megabytes}{150}

Maomao ได้รับมอบหมายให้ปรุงยาที่มีสรรพคุณ $M$ อย่าง ทำให้ Maomao จึงออกไปเตรียมยาที่ห้องปรุงยา แต่แล้วก็เกิดปัญหาขึ้น เมื่อพบว่าไม่มีสมุนไพรได้เหลืออยู่เลย ดังนั้น Maomao จึงต้องออกไปซื้อสมุนไพร 

\begin{center}
    \includegraphics[width=8cm]{kusuriya/Kusuriya-no-Hitorigoto-01-03-06.jpg}
\end{center}

ที่แหล่งซื้อขายสมุนไพรนั้นมีสมุนไพรอยู่ทั้งหมด $N$ ชนิด สมุนไพรชนิดที่ $i$ มีราคา $W_i$ โดย Maomao มีข้อมูลว่าสมุนไพรแต่ละชนิดมีสรรพคุณอะไรบ้าง โดยสำหรับสมุนไพรชนิดที่ $i$ และสรรพคุณชนิดที่ $j$ จะมีค่า $P_{i,j}$ ระบุว่าสมุนไพรชนิดที่ $i$ มีสรรพคุณชนิดที่ $j$ หรือไม่ โดยที่ $P_{i,j}=0$ คือสมุนไพรชนิดที่ $i$ ไม่มีสรรพคุณชนิดที่ $j$ และ $P_{i,j}=1$ คือสมุนไพรชนิดที่ $j$ มีสรรพคุณชนิดที่ที่ $j$

เหล่าผู้เข้าแข่งขันอันเก่งกาจช่วย Maomao เลือกสมุนไพรที่จะซื้อเพื่อให้ยาที่ปรุงมีสรรพคุณครบท้ัง $M$ ชนิด โดยใช้จำนวนเงินน้อยที่สุดเท่าที่จะเป็นไปได้ รับประกันว่าทุกสรรพคุณมีสมุนไพรอย่างน้อย $1$ ชนิดที่มีสรรพคุณนั้น

\InputFile
ข้อมูลนำเข้ามีทั้งหมด $N+1$ บรรทัด

บรรทัดแรกประกอบด้วยจำนวนเต็ม $N$ และ $M$ แทนจำนวนสมุนไพร และจำนวนชนิดของสรรพคุณ $(1\leq N\leq 10^{5},1\leq M\leq 16)$

บรรทัดที่ $2$ ถึง $N+1$ แต่ละบรรทัดประกอบด้วยจำนวนเต็ม $M+1$ จำนวน คือ $W_i,P_{i,1},P_{i,2},\ldots,P_{i,N}$ โดยที่ $W_i$ แทนราคาของสมุนไพรชนิดที่ $i$ และ $P_{i,j}$ ระบุว่าสมุนไพรชนิดที่ $i$ มีสรรพคุณชนิดที่ $j$ หรือไม่ $(1\leq W_i\leq 10^9,P_{i,j}\in\{0,1\})$

\OutputFile
ตอบจำนวนเต็มเพียงหนึ่งตัว แทนจำนวนเงินที่น้อยที่สุดที่ใช้ซื้อสมุนไพรเพื่อใช้ในการปรุงยาที่มีสรรพคุณครบทุกชนิด

\Scoring
ชุดทดสอบจะถูกแบ่งเป็น 6 ชุด จะได้คะแนนในแต่ละชุดก็ต่อเมื่อโปรแกรมให้ผลลัพธ์ถูกต้องในชุดทดสอบย่อยทั้งหมด

\begin{description}

\item[ชุดที่ 1 (9 คะแนน)] จะมี $1\leq N\leq 20$
\item[ชุดที่ 2 (3 คะแนน)] จะมี $M=1$
\item[ชุดที่ 3 (4 คะแนน)] จะมีสมุนไพรอย่างน้อย $1$ ชนิดเสมอที่มีสรรพคุณครบทั้ง $M$ ชนิด
\item[ชุดที่ 4 (13 คะแนน)] จะมี $1\leq N\leq 100$
\item[ชุดที่ 5 (22 คะแนน)] จะมี $1\leq N\leq 1000$
\item[ชุดที่ 6 (31 คะแนน)] จะมี $1\leq M\leq 10$
\item[ชุดที่ 7 (19 คะแนน)] จะมี $1\leq M\leq 13$
\item[ชุดที่ 8 (49 คะแนน)] ไม่มีเงื่อนไขเพิ่มเติม

\end{description}

\Examples

\begin{example}
\exmp{5 3
1000 1 1 0
100 1 0 0
300 0 1 1
150 0 1 0
20 0 0 1
}{270
}%
\end{example}

\Note

Maomao สามารถเลือกสมุนไพรชนิดที่ $2,$ $4$ และ $5$ ซึ่งจะใช้จำนวนเงิน $270$ ซึ่งน้อยที่สุดที่เป็นไปได้

\end{problem}

\end{document}
