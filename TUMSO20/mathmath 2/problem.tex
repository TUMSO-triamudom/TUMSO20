\documentclass[11pt,a4paper]{article}

\newcommand{\tumsoTime}{09:00 น. - 12:00 น.}
\newcommand{\tumsoRound}{1}

\usepackage{../tumso}
\usepackage[table]{xcolor}

\begin{document}

\begin{problem}{Math Math 2}{standard input}{standard output}{1 seconds}{512 megabytes}{150}

\textbf{โจทย์ข้อนี้รองรับเฉพาะภาษา C++ และ Python เท่านั้น}

% เดี๋ยวแก้ต่อ :D

ในช่วงที่นักเรียนในคลาสเรียนคณิตศาสตร์ของแพะกำลังเบื่อหน่ายกับเนื้อหาที่ซ้ำซาก แพะจึงได้ตั้งโจทย์แปลก ๆ เพื่อให้นักเรียนแต่ละคนได้ประลองกัน โดยแพะมีลำดับขนาด $N$ คือ $A_0,A_2,\ldots,A_{N-1}$ นักเรียนสามารถถาม $\gcd$ ของ $2$ พจน์ในลำดับ หรือ $\mathrm{lcm}$ ของ $3$ พจน์ในลำดับ โดยถ้าใครสามารถบอกลำดับทั้ง $N$ พจน์กับแพะได้คนแรกจะเป็นผู้ชนะ และได้รับรางวัลไป ซึ่งนักเรียนในห้องจะได้ถามตามลำดับวนไปเรื่อย ๆ โดย Icy นักเรียนสุดเจ๋งที่อยากได้รางวัลอย่างมาก เขาได้เป็นคนแรกที่เริ่มถาม ด้วยความที่เขาอยากชนะอย่างมาก จึงจะไม่ยอมให้ใครตอบไปก่อนเขา ซึ่งข้อมูลของนักเรียนแต่ละคนที่ถาม แพะจะกระซิบบอกคำตอบของคำถามนั้นให้เฉพาะกับคนที่ถาม (นั่นคือจะสามารถรู้ได้เฉพาะข้อมูลจากคำถามที่ตนเองถาม) Icy จึงอยากให้คุณช่วยในการถามเพื่อให้เขาเป็นผู้ชนะ

โจทย์ข้อนี้เป็บแบบ ถาม-ตอบ คุณสามารถเรียกฟังก์ชันต่อไปนี้ในการทำงานได้

\begin{itemize}
    \item \verb|int init()| เป็นฟังก์ชันเพื่อถามหาขนาดของลำดับ โดยเราต้องเรียกฟังก์ชันนี้ก่อนเรียกใช้ฟังก์ชันอื่น ๆ
    \item \verb|long long GCD(int i,int j)| ฟังก์ชันนี้จะคืนค่า $\gcd(A_i,A_j)$ โดยที่ $i\neq j$
    \item \verb|long long LCM(int i,int j,int k)| ฟังก์ชันนี้จะคืนค่า $\mathrm{lcm}(A_i,A_j,A_k)$ โดยที่ $i\neq j\neq k$
    \item \verb|void answer(std::vector<int> ans)| ให้เรียกฟังก์ชันนี้เพื่อตอบคำถามว่าแต่ละพจน์ของลำดับเป็นเท่าใด เมื่อเรียกฟังก์ชันนี้แล้ว โปรแกรมจะหยุดทำงาน
\end{itemize}

สำหรับภาษา Python ให้ใช้คำสั่ง \verb|from __main__ import init, LCM, GCD, answer| โดยในการเรียกฟังก์ชัน \verb|answer| ให้ใช้ตัวแปรประเภท \verb|List|

รับประกันว่า $4\leq N\leq 10^5$ และ $1\leq A_i\leq 10^5$ สำหรับทุก ๆ $i=0,2,\ldots,N-1$

ให้ $\gcd(x,y)$ หมายถึง \textcolor{blue}{\href{https://en.wikipedia.org/wiki/Greatest_common_divisor}{greatest common divisor (GCD)}} ของ $x$ และ $y$ \\
และ $\mathrm{lcm}(x,y,z)$ หมายถึง \textcolor{blue}{\href{https://en.wikipedia.org/wiki/Least_common_multiple}{least common multiple (LCM)}} ของ $x$, $y$ และ $z$

โปรแกรมของคุณจะต้องติดต่อกับ library  โดยให้ \verb|#include "mathmath2.h"| ที่ต้นโปรแกรมและในตอนคอมไพล์ให้นำ mathmath.cpp ไปคอมไพล์ด้วย ห้ามโปรแกรมทำการอ่านเขียนเอง

\Scoring

ในแต่ละชุดทดสอบจะได้คะแนนตามตารางด้านล่าง

\begin{center}
    \begin{tabular}{|c|c|}
        \hline
        \textbf{เงื่อนไข} & \textbf{อัตราส่วนคะแนนต่อคะแนนเต็มของปัญหาย่อยนั้น ๆ} \\
        \hline
        $S'\leq S$ & $1$ \\
        \hline
        \rule{0pt}{2pt}\centering $S'>S$ & $0.99\cdot(\frac{S}{S'})^{1.4}$ \\
        \hline
    \end{tabular}
\end{center}

เมื่อ $S'$ คือจำนวนครั้งที่เรียกฟังก์ชัน และ $S$ คือจำนวนครั้งที่ผู้แต่งโจทย์ทำได้

ชุดทดสอบจะถูกแบ่งเป็น 3 ชุด และคะแนนในแต่ละชุดคือคะแนนที่\textbf{น้อยที่สุด}จากทุกชุดทดสอบในชุดนั้น

\begin{description}

\item[ชุดที่ 1 (50 คะแนน)] จะมี $1\leq A_i\leq 20$
\item[ชุดที่ 2 (50 คะแนน)] จะมี $1\leq A_i\leq 1000$
\item[ชุดที่ 3 (50 คะแนน)] ไม่มีเงื่อนไขเพิ่มเติม

\end{description}

\section*{ตัวอย่าง}

กำหนดให้ลำดับมีขนาด $4$ ได้แก่ $3$, $4$, $5$, $6$

คุณเรียก \verb|init(N)| ฟังก์ชันนี้จะคืนค่า \verb|4| ผ่าน \verb|N| จะได้ว่า \verb|N=4|

คุณเรียก \verb|gcd(0,3)| ฟังก์ชันนี้จะคืนค่า \verb|3|

คุณเรียก \verb|gcd(1,3)| ฟังก์ชันนี้จะคืนค่า \verb|2|

คุณเรียก \verb|lcm(0,1,2)| ฟังก์ชันนี้จะคืนค่า \verb|60|

คุณเรียก \verb|answer({3,4,5,6})| โปรแกรมจะหยุดทำงาน

\end{problem}

\end{document}
