\documentclass[11pt,a4paper]{article}

\newcommand{\tumsoTime}{09:00 น. - 12:00 น.}
\newcommand{\tumsoRound}{1}

\usepackage{../tumso}

\begin{document}

\begin{problem}{Disaster 2}{standard input}{standard output}{1.5 seconds}{1024 megabytes}{250}

มีเมืองอยู่ $N$ เมือง ระบุด้วยหมายเลข $1$ ถึง $N$ เชื่อมกันด้วยถนนสองทิศทาง $N-1$ เส้น โดยถนนเส้นที่ $i$ จะเชื่อมระหว่างเมืองที่ $i$ กับ $i+1$ และมีระยะทาง $W_i$ หน่วย นอกจากนี้เมืองที่ $i$ จะมีประชากรอยู่ $P_i$ คน

เนื่องจากกำลังจะเกิดภัยพิบัติขึ้น จึงต้องการเลือกสร้างศูนย์อพยพในเมืองจำนวน $M$ เมือง โดยเมืองที่ $i$ จะมีค่าก่อสร้างศูนย์อพยพ $C_i$ บาท เมื่อเลือกแล้วประชากรทุกคนจะเดินทางไปยังศูนย์อพยพที่ใกล้เมืองที่ตนอยู่มากที่สุดศูนย์ใดก็ได้ และทางรัฐบาลจะต้องจ่ายเงินให้กับประชาชนแต่ละคนเป็นงบในการเดินทาง $1$ บาทต่อระยะทาง $1$ หน่วย คุณผู้เป็นนักวางแผนอัจฉริยะจึงถูกวานให้หาวิธีที่จะสร้างศูนย์อพยพในเมือง $M$ เมือง และทำให้จำนวนงบประมาณที่ทางรัฐบาลต้องเตรียมน้อยที่สุด

กล่าวคือ ให้ $S\in\mathcal{P}(\{1,2,\dots,N\})$ แทนเซ็ตของเมืองที่จะสร้างศูนย์อพยพ และ $f\colon\mathcal{P}(\{1,2,\dots,N\})\to\mathbb{Z}_{\ge 0}$ แทนจำนวนงบประมาณที่จะต้องเตรียม จะได้ว่า

$$f(S)=\sum\limits_{i\in S}C_i+\sum\limits_{1\leq i\leq N}P_i\cdot\min_{j\in S}d(i,j)$$

เมื่อ $d(i,j)$ แทนระยะทางระหว่างเมืองที่ $i$ และ $j$ จงหา $\min\limits_{S\colon |S|=M}f(S)$

\InputFile
ข้อมูลนำเข้ามีทั้งหมด $Q+2$ บรรทัด

บรรทัดแรกประกอบด้วยจำนวนเต็ม $N$ และ $M$ แทนจำนวนเมือง และจำนวนเมืองที่เลือกสร้างศูนย์อพยพ $(2\leq M\leq N\leq 10^5)$

บรรทัดที่ $2$ ประกอบด้วยจำนวนเต็ม $N-1$ จำนวน คือ $W_1,W_2,\ldots,W_{N-1}$ โดยที่ $W_i$ แทนระยะทางระหว่างเมือง $i$ กับ $i+1$ $(1\leq W_i\leq 10^3)$

บรรทัดที่ $2$ ประกอบด้วยจำนวนเต็ม $N$ จำนวน คือ $P_1,P_2,\ldots,P_N$ โดยที่ $P_i$ แทนจำนวนประชากรในเมืองที่ $i$ $(1\leq P_i\leq 10^3)$

บรรทัดที่ $2$ ประกอบด้วยจำนวนเต็ม $N$ จำนวน คือ $C_1,C_2,\ldots,C_N$ โดยที่ $C_i$ แทนจำนวนงบประมาณที่ต้องใช้ในการสร้างเมืองที่ $i$ $(1\leq C_i\leq 10^9)$

\OutputFile
ตอบจำนวนเต็มเพียงหนึ่งตัว แทนจำนวนงบประมาณที่ทางรัฐบาลต้องเตรียมที่น้อยที่สุด

\Scoring
ชุดทดสอบจะถูกแบ่งเป็น 7 ชุด จะได้คะแนนในแต่ละชุดก็ต่อเมื่อโปรแกรมให้ผลลัพธ์ถูกต้องในชุดทดสอบย่อยทั้งหมด

\begin{description}

\item[ชุดที่ 1 (7 คะแนน)] จะมี $2\leq N\leq 10$
\item[ชุดที่ 2 (18 คะแนน)] จะมี $2\leq N\leq 20$
\item[ชุดที่ 3 (35 คะแนน)] จะมี $2\leq N\leq 50$
\item[ชุดที่ 4 (27 คะแนน)] จะมี $2\leq N\leq 500$
\item[ชุดที่ 5 (69 คะแนน)] จะมี $2\leq N\leq 5000$
\item[ชุดที่ 6 (85 คะแนน)] จะมี $C_i=0$
\item[ชุดที่ 7 (9 คะแนน)] ไม่มีเงื่อนไขเพิ่มเติม

\end{description}

\Examples

\begin{example}
\exmp{5 2
1 1 1 1
1 2 3 4 5
2 4 6 8 10
}{20
}%
\end{example}

\Note

ถ้า $S=\{1,4\}$ จะได้ว่า \[f(S)=\sum\limits_{i\in S}C_i+\sum\limits_{1\leq i\leq N}P_i\cdot\min_{j\in S}d(i,j)=10+2+3+5=20\] ซึ่งเป็นค่าที่น้อยที่สุด

\end{problem}

\end{document}
